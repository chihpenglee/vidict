\input tex2page
\input btxmac

% Last modified 2004-03-26 

\title{{\tt\bf opted.vim}, 
a Vim plugin for consulting
Webster's Unabridged Dictionary (1913)}

\smallskip

\centerline{\urlh{../index.http}{Dorai Sitaram}}
\centerline{\urlh{vidict.tar.gz}{Download}}

\bigskip

The Online Plain Text English Dictionary, or
OPTED~\cite{opted}, can be consulted from the text
editor Vim~\cite{vim} quite easily.  The OPTED is a
public-domain dictionary based on the etext compiled by
Project Gutenberg~\cite{projgutenberg} from the
1913 version of Webster's Unabridged Dictionary.  It
has about 180,000 entries.

\subsection*{Installation}

Download the
\urlh{http://www.mso.anu.edu.au/~ralph/OPTED/v003.zip}{zip
file}
from the OPTED website and unzip
it.  (Use \p{unzip -aa} or other means to ensure that the text files
contain the correct newline format for your machine.)
This produces a directory \p{v003} containing many HTML
files.  Since \p{v003} is not a distinctive name, you
may want to rename the directory.  We will refer
to this directory as the OPTED directory, whatever its
actual name.

Put the file \p{opted.vim} in one of your Vim
\p{plugin} directories, e.g., \p{~/.vim/plugin}.
In your \p{.vimrc} or in a plugin file, let the global
variable \p{g:opted_dir} contain 
the
location of the OPTED directory, e.g.,

\p{
let g:opted_dir = '/home/ds26/opted/v003'
}

If you don't explicitly set \p{g:opted_dir} yourself, it will be assumed
to be \p{~/v003}.  I.e., the plugin will assume that you unpacked the
OPTED zip file in your home directory and didn't rename it. 

\subsection*{Usage}

From Vim, you may now use the command \p{gm} to open up 
a small (5-line) Vim window into the OPTED.

The OPTED window as seen in Vim is the result of
massaging the original OPTED files into Vim-suitable
\p{.txt} and \p{tags} files.  This massaging is a
time-consuming process, but it need be done only once,
and thus happens only the first time
\p{gm} is called.
(To avoid an unseemly delay when 
you try to consult the dictionary for the very first time
from Vim, you could pre-build the \p{.txt} and \p{tags}
files by calling \p{vim -c "norm gm" -c q} from your
operating system's
command line.)

Once in the OPTED window, you may use any of Vim's
several tags commands to specify a dictionary headword
as a tag.  E.g.,
\p{
:tj lexicon
}

or

\p{
:tj /lexicon
}

The Vim command \p{:help tags} shows you the various
ways to do tag search in Vim.  Vim allows both
completion and regular expressions on tags, so you don't have to
have the spelling down pat.  Jumping to the tag
will show the definition of the headword.  You can also
browse around the thus-obtained definition using normal
Vim navigation commands.

\subsection*{References}

\bibliographystyle{plain}
\bibliography{../tex2page/bigbib}

\bye

